\section{Лекция (Вводная)}
\subsection{Введение}
Эрлих  Иван Генрихович

Структура курса:
Основы вероятности и теория меры
ТВ (контрольная работа) + ДА (коллоквиум)

\subsection{Теория Вероятностей}

Предмет: случайный эксперимент.

Пример к чему можно применить этот предмет. Какова вероятность встретить динозавра на улице?) (Типичная жена программиста скажет 50\% --- либо встретишь либо нет)

\begin{enumerate}
	\item Отсутствие детерминированного результата и повторяемость
	
	Пример: Есть студент, корректен ли вопрос "Какова вероятность того что студент А получит отл 10 на экзамене по теории меры?" 
	Ответ: нет, вопрос безграмотен. Студент А уникален и будет сдавать максимум 3 раза, поэтому о повторяемости не может идти речи. Так же студент А будет меняться, приобретая знания, поэтому и о детерминированном результате говорить не приходится.
	
	\item Статистическая устойчивость
	
	% Как люди могут понять эту фразу $\lim_{N\to\infty} \frac{N(A)}{N} \exists \ldots $ 
	Примерная суть определения:
	Если провести очень много экспериментов (например, $10^4$ попыток) 3 раза, то частоты в 3х экспериментах не должны <<сильно>> отличаться друг от друга. 
\end{enumerate}



\begin{table}[!ht]
	\caption{Математическая модель эксперимента}
	\begin{tabular}{p{0.5\textwidth}|p{0.5\textwidth}}
		Эксперимент & Модель \\
		\hline
		результат эксперимента & Элементарный исход $ \{\omega_i\} = \Omega $  \\
		множество благоприятных результатов & событие $A \subseteq \Omega$ 
		(множество событий обозначается так: $\{A\} = \F$) \\
		$\frac{N(A)}{N}$ --- частота события (N - количество экспериментов, N(A) - количество благоприятных результатов) & $p$ - вероятность \\
		\hline
	\end{tabular}
\end{table}

Лирическое отступление о качестве модели: модель хорошая, если ее результаты хорошо сходятся с реальными результатами.

\subsection{Дискретная Теория Вероятностей}
В чем заключается идея? 
$|\Omega| < \infty$\text{, поэтому }$|\F| = 2^{|\Omega|}$

/* Свойства вероятности :
\begin{enumerate}
	\item $\Pb(\Omega) = 1$
	\item $\Pb(A \sqcup B) = \Pb(A) + \Pb(B)$
\end{enumerate}
*/

Достаточно задать $\Pb$ на $\omega_i$
$$\Pb(A) = \sum_{\omega_i: \omega_i \in A}p(\omega_i)$$

\begin{note}
	Книга по теории вероятностей от Ширяева~А.~Н. --- библия теории вероятности. Ее невозможно читать, но полезно сверить какие-то конкретные вопросы.
\end{note}
\subsubsection{Классическая Модель}  
--- элементарные исходы равновероятны 
$$
\left\{
\begin{aligned}
	& p(\omega_i)=\ldots=p(\omega_n)\\
	& \sum_{i = 1}^{n} p(\omega_i) = 1\\
\end{aligned}
\right.
\Rightarrow
\Pb(A) = \sum_{\omega_i: \omega_i \in A}p(\omega_i) = \frac{|A|}{|\Omega|}
$$    
 \begin{example}
 	Круглый стол,  $N$ гостей. Какая вероятность  $\Pb(\text{A и B сядут рядом})$
 	Предлагается такая модель:
 	\begin{enumerate}
 		\item $\omega_i = (i_1, \ldots, i_n) \qquad 1 \leq i \leq N$
 		$$\Pb(\text{A рядом с B}) = \frac{N*2*(N-2)!}{N!} = \frac{2}{N-1}$$
 		Первого сажаем куда угодно ($N$ мест), у второго 2 места (справа или слева) и дальше умножаем на $(N-2)!$ --- количество оставшихся свободных мест
 		
 		\item $\omega_i = \{i_1\} \qquad 1 \leq i_1 \leq N-1$ (Нумерация с позиции куда сел А)
 		$$\Pb(\text{A рядом с B}) = \frac{2}{N-1}$$
 		Ура, ответы совпали!
 	 \end{enumerate}
 \end{example}

\subsubsection{Неоклассической Дискретная Модель}  
Схема испытаний Бернулли:

--- Серия независимых (в бытовом смысле, пока определения нет) испытаний с двумя исходами $\{0,1\}$. $p$ --- вероятность 1.

$\omega_i = (i_1, \ldots, i_n) i_j \in \{0, 1\}$

$p \neq 1/2$ --- Неоклассическая модель  
$$
p(\omega_i) = p^{i_1}(1-p)^{1-i_1}p^{i_2}(1 - p)^{1-i_2} = p^{\sum i_j} (1-p)^{n - \sum i_j}
$$
Вероятность что в кортеже из $n$ элементов будет $k$ единиц:
$$
P\{\text{k едениц}\} = \sum_{\omega_i: \sum i_j = k} p(\omega_i) = \sum_{\omega_i: \sum i_j = k} p^k (1 - p)^{n-k} = C_n^kp^k(1-p)^{n-k}
$$

\subsubsection{Геометрическая вероятность}  
$\Omega \subseteq \R^n$ --- не дискретная история. $A \subseteq \Omega$

"Кидается" точка на каком-то множестве. 

$P(A) = \cfrac{\mu(A)}{\mu(\Omega)}$  

Проблема: Что делать если $A$ не измеримо (по Жордану и Лебегу), как тогда с этим работать? То есть $A \in \F $ ?
