\section{Лайфхаки и частые ошибки}
\subsection{А что это за документ?}

Добрый день, читатель!

Тут будет небольшой гайд из лайфхаков по использованию теха. Эти лайфхаки позволят делать конспект быстрее и профессиональнее. 


Данная статья будет состоять из двух разделов:
\begin{enumerate}
	\item Посмотрим на прикольные лайфхаки, повышающие продуктивность теха. 
	\item Посмотрим на частые ошибки и узнаем, как их не допускать.
\end{enumerate}

В конце есть раздел с полезными ссылками.

\subsection{Сосбственно, лайфхаки}

\subsubsection{Лайфхаки для повышения скорости}
\begin{enumerate}
	\item Учитесь печатать быстрее, как скорость теха почти полностью зависит от этого. Скорости в 50wpm(слов в минуту) должно хватать для того, чтобы техать лекции онлайн(вместе с лектором).
	
	\item Достаточно часто в математическом тексте возникают часто используемые шаблоны, паттерны. Поэтому имеются прекрасные способы ввести свои команды в конспекте. Например, набирать такой текст: 
	
	$f: \mathbb{R} \to \mathbb{R}$ (\verb|f: \mathbb{R} \to \mathbb{R}|) 
	
	или же
	
	$f: \R\to\R$ (\verb|f: \R\to\R|).
	
	Казалось бы, эффект тот же самый, но писать гораздо приятнее. 
	Синтаксис для переопределения команд такого типа  
	
	\verb|\newcommand{command_name}{script_of_command}|
	Так что советуем добавить такого рода сокращения в свой <<preamble>> файл, это сильно вам поможет. 
	
	\item Скачайте TexStudio на ваш компьютер, если часто приходится компилировать. 
	В нем время компиляции этого гайда занимает 1 секунду, на overleaf - 10-40 секунд.
	
	\item Если overleaf шакалит в google chrome (на mac OS), используйте Safari. 
\end{enumerate}

\subsubsection{Лайфхаки для повышения качества}
В данном разделе я разберу некоторые полезные фичи из файла <<style.tex>>.
\begin{enumerate}
	\item Не кладите весь код в один .tex файл!
	Это делает код менее структурированным и сложным для понимания. 
	Четко разделяйте код на следющие файлы (как код в нашем официальном шаблоне, ссылка в конце гайда):
	\begin{itemize}
		\item main.tex файл, в который вставляется код из других файлов
		
		\item preamble.tex файл с преамбулой
		
		\item lectures/ папка с файлами отдельных лекций, например, "lecture05.tex" или просто "05.tex". Не техайте весь курс в одном файле!
		
		\item Отдельный файл с библиографией или просто дополнительной информацией 
		
		\item images/ папка с картинками (чтобы это сделать нужно будет явно указать путь до картинок командой \verb|\graphicspath{{images/}}|)
	\end{itemize}
	PS: Файлы вроде title\_page, preamble и additional удобно класть в отдельную папку (например, etc/).
	
	\item Часто возникает потребность в командах, которые требуют аргументов (как функции в большинстве языков программирования).
	Пожалуй, самый удачный пример такой комманды~--- команда <<brackets>> (Ее определение можно найти в preamble) и команды  \verb|\left, \right|, суть которых в автоматической подгонке размеров скобок под аргумент. (Примеры ниже)
	\begin{table*}[!ht]
		Примеры по использованию brackets:
		\begin{multicols}{2}
			$$
			\sum \brackets{\sum\limits_{k = 0}^n \brackets{e^{-\ln k + 2\ln\ln k}}}
			$$
			
			\columnbreak
			\begin{verbatim}
			\sum \brackets{\sum\limits_{k = 0}^n 
			\brackets{e^{-\ln k + 2\ln\ln k}}}
			\end{verbatim}
		\end{multicols}
		или
		\begin{multicols}{2}
			$$
			\sum (\sum\limits_{k = 0}^n (e^{-\ln k + 2\ln\ln k}))
			$$
			
			\columnbreak
			\begin{verbatim}
			\sum (\sum\limits_{k = 0}^n 
			(e^{-\ln k + 2\ln\ln k}))
			\end{verbatim}
		\end{multicols}
		
	\end{table*}
	
	
	\begin{table}[!ht]
		И еще пример (\verb|\left, \right| можно использовать с любыми скобками):
		$$
		\lim_{n \to \infty}P\brackets{\left|\frac{\sum\limits_{i = 1}^{n}\xi_i - \E\brackets{\sum\limits_{i = 1}^{n}\xi_i}}{n^{\frac{1}{2} + \delta}}\right| > \epsilon} = 0 \Longleftrightarrow 
		\lim_{n \to \infty}P(|\frac{\sum\limits_{i = 1}^{n}\xi_i - \E(\sum\limits_{i = 1}^{n}\xi_i)}{n^{\frac{1}{2} + \delta}}| > \epsilon) = 0
		$$
	\end{table}
	
	\item Часто в тексте используют стандартные <<заголовки>> для обозначения новой части повествования.
	Например, новая теорема, пример и прочее. 
	В связи с этим я предлагаю обратить внимание на окружения типа \verb|\begin{theorem}| или \verb|\begin{proof}|  и прочие (опять же в <<preamble>> заданы основные окружения). Примеры использования есть в официальном примере теха лекции (ссылка в конце документа). 
	
	\item Если у вас формула не влезает в одну строку, то почитайте про окружение <<multline>> вместо разрыва одной формулы на несколько строк.
	\begin{table}[!ht]
		\begin{multicols}{2}
			Очень длинная формула:
			\begin{multline}
			1 + 2 + 3 + 4 + 5 + \ldots + \\
			11 + 12 +13 + 14 + 15 + \ldots + \\ 
			99 + 100 = 5050
			\end{multline}
			
			\columnbreak
			
			\begin{verbatim}
			\begin{multline}
			1+2+3+4+5+\ldots + \\
			11+12+13+14+15+\ldots+\\ 
			99 + 100 = 5050
			\end{multline}
			\end{verbatim}
		\end{multicols}
	\end{table}
	
	Если же хочется несколько строчек отдельных формул –– то про окружение <<align>> и <<aligned>>.
	\begin{table}[!ht]
		\begin{multicols}{2}
			"Матрица" формул: 
			\begin{align}
			a &=  b + c + d \\
			c + d + 4 &= e + f + g
			\end{align}
			
			\columnbreak
			
			\begin{verbatim}
			\begin{align}
			a &=  b + c + d \\
			c + d + 4 &= + e + f + g
			\end{align}
			\end{verbatim}
		\end{multicols}
		\begin{multicols}{2}
			$$
			\left\{
			\begin{aligned}
			& x+1=1\\
			& \ldots\\
			& x + N = N
			\end{aligned}
			\right.
			$$    
			
			\columnbreak
			
			\begin{verbatim}
			$$ 
			\left\{
			\begin{aligned}
			& x+1=1\\
			& \ldots\\
			& x + N = N
			\end{aligned}
			\right. 
			$$
			\end{verbatim}
		\end{multicols}
	\end{table}
	
	\item Достаточно часто при конспектировании лекции может прийти озарение, как описать достаточно весомое доказательство теоремы русским языком так, что будет понятно любому.
	Поэтому не бойтесь приводить в конспектах свои аналогии некоторых фактов. 
	И да, сухой авторский слог конечно хорош, но иногда передать полностью красоту некоторого факта или же подготовить читателя к трудному логическому переходу куда уместнее с помощью <<юмористического комментария>>. 
	Это только рекомендация автора, не более и не менее.
	
	\item Иногда внутри длинной формулы необходимо вставить текст на русском языке.
	Для этого используйте команду <<text>>.
	
	\item Иногда не хочется, чтобы размер символов уменьшался в дроби или в возведении в степень, тогда можно воспользоваться универсальной командой \verb|\displaystyle | или \verb|\cfrac| для дроби. Пример ниже.
	\begin{table}[!ht]
		\begin{multicols}{2}
			\begin{align*}
			4 &= \frac{2 + \frac{8}{4}}{6} \\
			4 &= \frac{3 + \cfrac{4}{4}}{6}\\
			4 &= \frac{2 + \frac{\displaystyle 8}
				{\displaystyle 4}}{6}\\
			4 &= 2^{7 + v}\\
			4 &= 9^{\displaystyle7 + v}\\
			\end{align*}   
			
			\columnbreak
			
			\begin{verbatim}
			\begin{align*}
			4 &= \frac{2 + \frac{8}{4}}{6} \\
			4 &= \frac{3 + \cfrac{4}{4}}{6}\\
			4 &= \frac{2 + \frac{\displaystyle 8}
			{\displaystyle 4}}{6}\\
			4 &= 2^{7 + v}\\
			4 &= 9^{\displaystyle7 + v}\\
			\end{align*}   
			\end{verbatim}
		\end{multicols}
	\end{table}
\end{enumerate}

\subsection{Частые ошибки}

Опять же автор не берет на себя никаких обязательств, что эти ошибки встречаются часто, но позволю в силу своего скромного опыта их так обозвать.
\begin{enumerate}
	\item[0.] Грамматические, пунктуационные и прочие ошибки, связанные с великим и могучим, я не рискну разбирать. 
	Старайтесь писать так, чтобы вы сами могли распарсить. 
	Если при этом вам удалось передать мысль, не вызвав у читателя когнитивного диссонанса, то считайте, что вы уже большой молодец.
	
	Из общих советов~--- старайся не делать предложения длинной больше 2х строчек. К концу такого предложения мысль теряется, парсь текст на меньшие куски.
	
	\item В \LaTeX \ имеются автоматические отступы в окружениях (они используются, поскольку символы пробеллов удаляются после команд). 
	Старайся их выдерживать, тогда сурсы читабельнее будут. Сравните:
	
	\LaTeX \ is great! (\verb|\LaTeX \ is great!|) or \LaTeX is great! (\verb|\LaTeX is great!|)
	
	\item Если вы уж и используете \textit{курсив} (\verb|\textit{курсив}|) или \textbf{жирное} (\verb|\textbf{жирное}|) написание, то делайте это с умом. 
	И да, выделять слова КАПСОМ никогда не было приятно с визуальной точки зрения.
	\item Крайне рекомендую прочитать про неразрывные тире и дефисы в тексте. 
	После такого ваша жизнь вряд ли станет прежней.
	
	\item Все переменные в тексте окружайте с помощью знаков доллара. В частности, перед вами два варианта: -a, $-a$ (\verb|-a, $-a$|).
	
	\item Никогда не делайте перенос строки через два слэша, если это не формула. 
	Это нереально затрудняет чтение. С двойным переносом строки всегда приятнее читать сурсы, поверьте.
	\begin{verbatim}
	Читабельно:
	Первая Строка
	
	Вторая строка
	
	Менее читабельно:
	Первая Строка\\ Вторая строка
	\end{verbatim}  
	Хотя результат тот же.
	
	\item Многоточия не нужно писать тремя точками..., лучше командой \ldots (\verb|\ldots|)
	
	\item Игнорирование предупреждений и ошибок пока их не станет сотня, и вы не сможете даже скомпилировать документ. "Почему я не могу отформатировать файл? Я должен закончить главу 2 к завтрашнему дню."
	
	Поэтому  \textbf{совет} –– читать и стараться понимать build log! (Ссылка на список с основными типами ошипок в log в конце)
	Поиск источника ошибок компиляции довольно сложен для начинающих \LaTeX (иногда даже для более продвинутых пользователей), поэтому очень полезно искать соответствующие сообщения об ошибках!
	
	\item \textbf{НЕ использование системы контроля версий для вашего документа.}
	
	Если вы уверенны, что (вы или ваш напарник) нечаянно не сотрете половину документа. 
	Или что в какойто момент он не перестанет компилироваться, а вы даже не будете понимать, в чем именно проблема, то дерзайте \ldots 
	
	Так что github вам в помощь) И еще совет: не храните в гите бинарники вроде .pdf. У гита проблемы с изменением бинарников, и после 10 лекций у вас будет минимум 10 копий вашего документа в истории. 
	
	\item Использование font attribute commands или (old-style font commands), как если бы они принимали аргументы: This is {\bfseries{important} (\verb|\bfseries{important}|) message\ldots why is my entire dissertation bold?}
	
	\item \textbf{Огромные преамбулы}
	
	Иногда люди любят использвать огромные преамбулы, пожертвованные продвинутыми друзьями, не имея ни малейшего представления, для чего они нужны. В конце концов возникает загадочная несовместимость, меня просят помочь, и я понятия не имею, какие пакеты на самом деле используются. Автор тоже.
	
	Так что советую разбираться в преамбулах, особенно в преамбуле нашего шаблона. В нем только самое необходимое, она всего 100 строчек.
	
	\item Использование \verb|$$| для italic \verb|-_-|.  Тут без комментариев.
	
	\item \textbf{Читать туториалы, написанные в каменном веке.}
	
	Многие решения оттуда сейчас считаются плохими.
	
\end{enumerate}                         
