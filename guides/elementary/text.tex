\section{Введение}
\subsection{Что за документ?}

Добрый день, читатель!

Если ты читаешь этот гайд, скорее всего, имеешь недостаточно опыта теха за плечами. 
Здесь же ты найдешь все, чтобы начать и продолжить техать :).

\subsection{С чего начать?}

В первую очередь вам стоит задуматься о скорости печати, серьезно, в техе это \textit{ключевой навык}. Так что если вы печатаете 2мя пальцами или не владеете слепой пеечатью, читайте дальше.

Основные команды можно освоить за день, после нескольких затеханных конспектов вы их запомните, а вот на обучение беглой печати всеми пальцами потребуются время и регулярная практика.

Мне очень помог \href{http://klavogonki.ru/}{этот сайт}, в нем пользователи добавляют собственные словари (разные языки, разные группы слов), другие ими пользуются. 
Изучение происходит в формате `гонок', вы на скорость печатаете текст или просто набор слов, а потом следите за своим прогрессом и постепенно увеличиваете скорость.

Также неплох \href{https://www.ratatype.com}{этот тренажер}.

Тренируйтесь как на русской, так и на английской раскладке.

\subsection{А как техать?}
Для быстрого старта в техе советую пройти бесплатный \href{https://www.coursera.org/learn/latex}{курс от ВШЭ}. Первые две части можно освоить приблизительно за 4-5 часов, на прохождение всего курса достаточно двух выходных дней. В курсе разобраны основные команды, есть ссылки на полезные сервисы, доступен весь исходный тех. После него даже если у вас и останутся вопросы, они будут решаться редким гуглением или ответами товарищей. 
\\
(В нем можно не смотреть часть про библиографию в 4й неделе. А из 5й недели достаточно посмотреть только видео "Большие проекты".)

Есть так же книги по теху, сымые известные из них \href{https://drive.google.com/file/d/1m3SDxmCL_f_X4J-IM2F_Wde0ez6ke0bx/view?usp=sharing}{Львовского} в 1200стр или \href{https://drive.google.com/file/d/140a3M7XQWOEBac07WjqbXdJ3xjHzieGM/view?usp=sharing}{Воронцова} в 60стр. Но, если честно, их читать не так приятно и быстро, как смотреть курс на coursera.

\subsection{Где техать}
Примерно это-же сказанно в курсе от Вышки, но мы продублируем.
Есть два пути:
\begin{enumerate}
    \item Пользоваться онлайн компилятором, вроде \href{https://www.overleaf.com/}{overleaf}.
    
    Преимущества: 
    \begin{itemize}
        \item Самый простой путь. Нужно просто зарегистрироваться (например, через гугл) 
        
        \item Можно совместно работать над проектом в real-time.
    \end{itemize}
    \item Установить компилятор на свой компьютер 
    
    Преимущества: 
    \begin{itemize}
        \item Скорость. Придется немного запариться в начале (совсем чуть-чуть) но вы скорость компиляции выйдет на новый уровень (1с вместо 5-40с в онлайн компиляторе) 
    \end{itemize}
\end{enumerate}

\subsection{Установка}
Чтобы использовать систему LaTeX на компьютере, нужно, по крайней мере, установить две вещи: дистрибутив LaTeX и программу для редактирования кода (иногда она идет вместе с дистрибутивом).
Современные дистрибутивы LaTeX — большие комплексы программ и пакетов, занимающие несколько гигабайт, их скачивание и установка могут занять продолжительное время.


Более подробную информацию по установке вы найдете в начале курса.

\subsubsection{Windows}
Ссылку на установку файла install-tl.exe (12 MB) можно найти \href{http://www.tug.org/texlive/acquire-netinstall.html}{на сайте TUG}. 
Запустите этот файл, чтобы скачать и установить систему TeX Live. 
Настройки, предлагаемые по умолчанию во время установки, можно не менять. 
Вы можете использовать любую программу для редактирования кода. 
Рекомендованную нами программу-редактор TeXstudio можно скачать с \href{http://texstudio.sourceforge.net/#download}{официального сайта}.

\subsubsection{OS X}
Если вам не жалко 2.5k на удобный редактор, зайдите в App~Store и скачайте Texpad:~LaTeX~editor. Все)

Лучшим бесплатным вариантом является TeXstudio.
Установите текущую версию дистрибутива\href{https://tug.org/mactex/}{MacTeX}. 
Нужно скачать файл MacTeX.pkg и запустить его на компьютере. 
Вместе с системой TeX Live на ваш компьютер будут установлены некоторые полезные программы, в том числе редактор TeXworks. 
Вы можете использовать любую программу для редактирования кода. 
Рекомендованную нами программу \href{https://www.texstudio.org}{TeXstudio} можно скачать с официального сайта (у нее старомодный интерфейс, но ничего лучше пока нет, и она работает хорошо).

\subsubsection{Linux (Debian/Ubuntu)}

Чтобы установить систему TeX Live , введите в терминале:
\begin{lstlisting}[backgroundcolor = \color{light-gray}, language=bash]
sudo apt-get install texlive-full
\end{lstlisting}

Установите пакет texlive-latex3 (включает, в частности, LaTeX-пакет mathtools):
\begin{lstlisting}[backgroundcolor = \color{light-gray}, language=bash]
sudo apt-get install texlive-latex3
\end{lstlisting}


Для работы с русским языком выполните:
\begin{lstlisting}[backgroundcolor = \color{light-gray}, language=bash]
sudo apt-get install texlive-lang-cyrillic
\end{lstlisting} 


Еще пара пакетов:
\begin{lstlisting}[backgroundcolor = \color{light-gray}, language=bash]
sudo apt-get install texlive-latex-extra
\end{lstlisting}

*Чтобы избавить себя от необходимости последующей установки ряда недостающих пакетов, можно сразу же установить полную версию TeX Live:
\begin{lstlisting}[backgroundcolor = \color{light-gray}, language=bash]
sudo apt-get install texlive-full
\end{lstlisting}

Вы можете использовать любую программу для редактирования кода. Для установки последней версии рекомендованной в данном курсе программы TeXstudio можно скачать соответствующий пакет с \href{http://texstudio.sourceforge.net/#download}{официального сайта}.

\subsection{Что дальше?}
\begin{enumerate}
    \item После курса от ВШЭ, нужно прочитать \href{https://drive.google.com/file/d/1ItSd7wIKDC0uJiJ-Hifd1JO3MKASVVKl/view?usp=sharing}{гайд для продолжающих}. 
    В нем разобрано много полезных лайфхаков и упомянуты частые ошибки(и их пути их решения). 
    
    \item От себя хочу добавить, что на youtube есть просмотр автоматически сгенерированных субтитров лекции. 
    Они далеко не идеальны, но могут кому-то помочь в текстовом наполнении конспекта.

\end{enumerate}
